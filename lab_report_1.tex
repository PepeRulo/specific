%%%%%%%%%%%%%%%%%%%%%%%%%%%%%%%%%%%%%%%%%
% University/School Laboratory Report
% LaTeX Template
% Version 3.1 (25/3/14)
%
% This template has been downloaded from:
% http://www.LaTeXTemplates.com
%
% Original author:
% Linux and Unix Users Group at Virginia Tech Wiki 
% (https://vtluug.org/wiki/Example_LaTeX_chem_lab_report)
%
% License:
% CC BY-NC-SA 3.0 (http://creativecommons.org/licenses/by-nc-sa/3.0/)
%
%%%%%%%%%%%%%%%%%%%%%%%%%%%%%%%%%%%%%%%%%

%----------------------------------------------------------------------------------------
%	PACKAGES AND DOCUMENT CONFIGURATIONS
%----------------------------------------------------------------------------------------

\documentclass{article}

\usepackage[version=3]{mhchem} % Package for chemical equation typesetting
\usepackage{siunitx} % Provides the \SI{}{} and \si{} command for typesetting SI units
\usepackage{graphicx} % Required for the inclusion of images
\usepackage{natbib} % Required to change bibliography style to APA
\usepackage{amsmath} % Required for some math elements 

\setlength\parindent{0pt} % Removes all indentation from paragraphs

\renewcommand{\labelenumi}{\alph{enumi}.} % Make numbering in the enumerate environment by letter rather than number (e.g. section 6)

\usepackage{pgfplots}
\usepackage[margin=0.75in, paperwidth=8.5in, paperheight=11in]{geometry}
\usepackage{setspace}
\usepackage{booktabs}
\usepackage[]{units}


% For nicely typeset tabular material
\usepackage{booktabs}

%%
% For graphics / images
\usepackage{graphicx}
\setkeys{Gin}{width=\linewidth,totalheight=\textheight,keepaspectratio}
\graphicspath{{graphics/}}

%%
% Additional
\usepackage{units}
\usepackage{amsmath,amsfonts,amsthm} % Math packages
\usepackage{mathtools}% http://ctan.org/pkg/mathtools
%\usepackage{mparhack}
\usepackage{sectsty} % Allows customizing section commands
%\usepackage[dvipsnames]{xcolor}
\usepackage{pgf,tikz}
%\usepackage{pgfplots}
%\usetikzlibrary{shapes,arrows}
%\usetikzlibrary{patterns,fadings}
%\usetikzlibrary{arrows}
% \usetikzlibrary{decorations.pathreplacing}
% \usetikzlibrary{snakes}
 %\usetikzlibrary{spy}
 %\usepackage{setspace}
% \usepackage{3dplot}
% \usepackage{cancel}
%\usepackage{physymb}
%\usepackage{braket}
%\usepackage{verbatim}
%\usepackage[x11names]{xcolor}                     %Additional colors
%\usepackage{euler}  



% The fancyvrb package lets us customize the formatting of verbatim
% environments.  We use a slightly smaller font.
\usepackage{fancyvrb}
\fvset{fontsize=\normalsize}

%\usepackage{times} % Uncomment to use the Times New Roman font

%----------------------------------------------------------------------------------------
%	DOCUMENT INFORMATION
%----------------------------------------------------------------------------------------

\title{Specific Heat Capacity of Metals \\ PHYS 442} % Title

\author{Jose Raul Pastrana } % Author name

\date{\today} % Date for the report

\begin{document}

\maketitle % Insert the title, author and date

\begin{center}
\begin{tabular}{l r}
Date Performed: & November 10th, 2015 \\ % Date the experiment was performed
Partners: & Whole class \\ % Partner names
Instructor: & Dr. Schultz % Instructor/supervisor
\end{tabular}
\end{center}

% If you wish to include an abstract, uncomment the lines below
% \begin{abstract}
% Abstract text
% \end{abstract}

%----------------------------------------------------------------------------------------
%	SECTION 1
%----------------------------------------------------------------------------------------

\section{Objective}

The objective of this experiment is to measure the specific heat capacity of three different samples of unknown metals and to compare those with the accepted values.  \\ 

\section{Definitions}
\label{definitions}
\begin{description}
\item[Heat]
Heat is the measure of the internal kinetic energy of a substance.
\item[Temperature]
Temperature is a measure of the kinetic energy of a particle.  It is the degree or intensity of heat in a substance.  Celcius is a unit of temperature.  One degree Celcius represents the temperature change of one gram of water when $2.39\times10^{-5}\text{Joules}$ of heat is added to it.

\item[Specific Heat Capacity]
The specific heat capacity is the energy transferred to one kilogram of substance causing its temperature to increase by one degree Celcius. \cite{Homer:2014}

\item[Thermal Equilibrium]
Thermal equilibrium is a condition where two substances in physical contact with each other exchange no net heat energy.  Substances in thermal equilibrium are at the same temperature.
\end{description}


\section{Theory}
The change in the internal energy of an object or substance is equal to the product of the mass and the specific heat capacity and the change in temperature.
$$\Delta U=mC_p\Delta T$$
When water and the metal samples are in thermal equilibrium the change in heat of the water is equal in magnitude to the change in heat of the metal.
$$\Delta U_{metal}=\Delta U_{water}$$
From this relationship we may derive a formula for the specific heat capacity of the metal sample given the mass of metal, mass of water, change in temperature of the water, change in temperature of the metal and the specific heat capacity of water.
$$m_{metal}C_{metal}\Delta T_{metal}=m_{water}C_{water}\Delta T_{water}$$\\
\\
$$\boxed{C_{metal}=\frac{m_{water}}{m_{metal}}  \frac{\Delta T_{water}}{\Delta T_{metal}}    C_{water}}$$

\newpage

\section{Materials}
\begin{itemize}
\item kettle 
\item unknown metal samples (aluminum, zinc and copper)
\item styrofoam cups
\item graduated cylinder
\item scale
\item thermometer
\item tongs
\item flask of water
\end{itemize}

\begin{figure}[h]
\begin{center}
\includegraphics[width=0.8\textwidth]{pic} % Include the image placeholder.png
\caption{Experimental materials}
\end{center}
\end{figure}

\section{Method}

\begin{enumerate}
\item Weigh the samples and record
\item Measure 350 ml of water in graduated cylinder and transfer to styrofoam cup
\item Measure the initial temperature of the water
\item Boil water and add metal samples to kettle
\item Use tongs to transfer a sample to the cup with water
\item Place thermometer in cup, cover it, stir and record equilibrium temperature
\item Repeat steps b-f for each sample. For the last sample (64.1 g one) measure 300 ml of water. 

\end{enumerate}

\newpage

\section{Data}

\begin{table}[htbp]
\begin{center}
\footnotesize
\begin{tabular}{lllll}
\toprule
 Metal   & Mass Metal & Mass Water & Temp Water Initial & Temp Final \\                                                      
\midrule
  
    Cube (Al)   & 90.5 g  &350 g         & 20.5 Celcius & 24.5 Celcius   \\
    Long cylinder (Cu)     & 203.0 g    & 350 g       & 20.8 Celcius & 24.8 Celcius  \\
    Short cylinder (Zn)   & 64.1 g    &300 g       & 20.9 Celcius  & 22.5 Celcius  \\
    
\bottomrule
\end{tabular}
\end{center}
  \caption{Experimental data}
  \label{tab:font-sizes}
\end{table}

\begin{table}[htbp]
\begin{center}
\footnotesize
\begin{tabular}{lc}
\toprule
 Material  & Specific Heat Capacity \\                                                      
\midrule
  
    Water   & $4180\  \nicefrac{ \text{J}}{\text{kg}\cdot {}^{\circ}\text{C}}$       \\
    Aluminum   & $900\   \nicefrac{ \text{J}}{\text{kg}\cdot {}^{\circ}\text{C}}$       \\
    Copper     & $387\   \nicefrac{ \text{J}}{\text{kg}\cdot {}^{\circ}\text{C}}$      \\
    Zinc     & $380\  \nicefrac{ \text{J}}{\text{kg}\cdot {}^{\circ}\text{C}}$      \\
     Iron     & $452\   \nicefrac{ \text{J}}{\text{kg}\cdot {}^{\circ}\text{C}}$      \\
     Steel     & $452\   \nicefrac{ \text{J}}{\text{kg}\cdot {}^{\circ}\text{C}}$      \\
     Lead     & $128\   \nicefrac{ \text{J}}{\text{kg}\cdot {}^{\circ}\text{C}}$      \\
     Silver     & $230\   \nicefrac{ \text{J}}{\text{kg}\cdot {}^{\circ}\text{C}}$      \\
    
\bottomrule
\end{tabular}
\end{center}
  \caption{Known specific heat capacities}
  \label{tab:font-sizes}
\end{table}

\newpage

\section{Example Calculations}
This is the calculation for the specific heat capacity of the long cylinder (copper).
$$C_{metal}=\frac{m_{water}}{m_{metal}}  \frac{\Delta T_{water}}{\Delta T_{metal}}    C_{water}$$
$$\Delta T_{water}=24.8-20.8=4.0 {}^{\circ}\text{C}$$
$$\Delta T_{metal}=100-24.8=75.2 {}^{\circ}\text{C}$$
$$C_{metal}=\frac{0.350 \text{kg}}{0.203 \text{kg}}  \frac{4.0  {}^{\circ}\text{C}}{75.2  {}^{\circ}\text{C}}    4180\  \nicefrac{ \text{J}}{\text{kg} {}^{\circ}\text{C}}=383\  \nicefrac{ \text{J}}{\text{kg}\cdot {}^{\circ}\text{C}}$$

The percent error is calculated as follows.
$$Error=\frac{387-383}{387}=1.0\%$$



\section{Results}

\begin{table}[htbp]
\begin{center}
\footnotesize
\begin{tabular}{llll}
\toprule
 Metal  & Material  & Measured $C_p$ & Percent Error\\                                                      
\midrule
  
    Cube    & Aluminum   & $856\   \nicefrac{ \text{J}}{\text{kg}\cdot {}^{\circ}\text{C}}$ &   4.8\%   \\
    Long cylinder   & Copper     & $383\   \nicefrac{ \text{J}}{\text{kg}\cdot {}^{\circ}\text{C}}$   & 1.0\%   \\
    Short cylinder  & Zinc     & $404\  \nicefrac{ \text{J}}{\text{kg}\cdot {}^{\circ}\text{C}}$  & 6.3\%   \\
    
\bottomrule
\end{tabular}
\end{center}
  \caption{Calculated specific heat capacities}
  \label{tab:font-sizes}
\end{table}


 
%----------------------------------------------------------------------------------------
%	SECTION 2
%----------------------------------------------------------------------------------------

\section{Discussion of Error}
Possible sources of error in this lab include the following. 
The main source of error is heat loss, as heat was released from the thermally isolated system within the styrofoam cups when opened to measure the metals' temperature. During this brief moment heat was lost, although minimally. Also –although a really thermally protected environment given by two styrofoam cups, which are of an excellent material to prevent the heat from dissipating, covered by another two cups– heat could have been absorbed by the cups, resulting in a less accurate measurement (infinitesimally).  
Another source of error is the purity of the metals. Although they were aluminum, copper and zinc, purity plays a key point in determining their heat capacity. If one of the metals wasn't pure enough (as in containing part of another metals (being a blend to minimize costs)), its specific heat capacity varied from the real one. 

\section{Conclusion}

During the experiment there was a crucial obstacle: determining what metal was the short cylinder (later found to be zinc), since its specific heat capacity was close to that of iron and steel, copper, and zinc. Through its color (gray), however, the possibility of it being copper was discarded. After this it was a mater of perceptual analysis resolving that it couldn't be neither steel nor iron. Moreover, its capacity was much closer to that of zinc than to these two. 

After determining all of the samples' specific heat capacities, they were found to be aluminum, zinc and copper. This was later confirmed by the professor. In this way, it can be seen how specific heat capacity can be used practically to determine the identity of a given material. 


\bibliographystyle{apalike}

\bibliography{sample}

%----------------------------------------------------------------------------------------


\end{document}